\section{Related work\textsuperscript{2}}
Gao et al. \cite{8338177} proposed a platform to let electric vehicles (EVs) participate in Vehicle-to-Grid (V2G) networks for bidirectional electric energy transmission and payment settlement for consumed energy. They argued that their proposed system would allow valuable statistical information for load forecasting, price prediction or optimal energy consumption scheduling. Their concept was implemented with Hyperledger Fabric and they also observed the significantly increased transaction throughput, a permissioned blockchain approach can provide, compared to public blockchains such as Bitcoin or \\ Ethereum.

Lundqvist et al. \cite{8016254} realized a \textit{smart cable}, which pays the \textit{smart socket} counterpart for electricity consumption via Bitcoin. They also identified transaction fees as the main problem for micro-payment use cases, which they mitigated by batching of transactions. They concluded that this approach implies, that with increasing transaction batch-size for reducing relative proportion of fees, the risk for fraud equally increases.  

Novo \cite{8306880} proposed a concept based on a private \\ Ethereum blockchain for communication and access management of IoT devices. Each device communicates with the so called \textit{management hub} and  does not interact directly with the blockchain. They conclude that with the rapidly rising number of IoT devices connected to the internet, centralized management systems would not be able to handle the transaction quantity and minimize reliability by forming a single point of failure. They also observed, that computational constrained IoT devices would need an intermediary layer to connect to blockchain platforms because of storage space limitations.

Strugar et al. \cite{DBLP:journals/corr/abs-1804-00658} followed a machine-to-machine economy driven approach, by envisioning a charging architecture and billing system for Electric Autonomous Vehicles (EAVs) using the IOTA cryptocurrency for payments, avoiding the occurrence of transaction fees. They emphasized on the early stage of IOTA development and existing security concerns, regarding the cryptographic implementations, not following best practices. Albeit, the approach was convincing, their work was theoretical without an proof-of-concept implementation.

With PlaTIBART, Walker et al. \cite{PlaTIBART} proposed a platform for development, deployment, execution, management and testing of IoT blockchain applications. They criticized the lack of a fault tolerant, secure Ethereum client with an emphasize to IoT devices and pointed out the necessity for such a platform through the large number of attacks and hacks on Ethereum smart contracts, where Ether worth millions of Dollars got stolen.

Unlike this paper, none of the introduced papers emphasized on a payment and management system where different parties have a profit-sharing business model as its common with pay-per-use systems.

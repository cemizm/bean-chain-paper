\section{Introduction\textsuperscript{1}}
According to the Gartner Hype Cycle 2017 for emerging technologies, blockchain has just passed the peak of inflated expectations \cite{TopTrend28:online}. While the technology itself enables the creation of decentralized applications with immutable and tamper proof transactions without the need of a controlling third party, it is currently primarily used in FinTech solutions or namely in cryptographic currencies. The peak of the hype cycle is associated with more than 1,600 different cryptocurrencies with a market capitalization of around 350 billion dollars \cite{AllCrypt83:online}, and in contrast only a relatively small number of proof of concepts for other applications \cite{8338177,8016254,8306880,DBLP:journals/corr/abs-1804-00658,PlaTIBART}. To overcome the phase of disillusionment and reach the plateau of productivity, useful mainstream and enterprise applications are required. This can be accomplished by transforming existing applications to create more efficient, secure and cost-effective solutions based on blockchain technologies \cite{REYNA2018173}. At the same time, the blockchain technology creates opportunities for new applications that were previously difficult or impossible to implement in a transparent, decentralized environment, such as tracking goods in a supply chain \cite{doi:10.1002/isaf.1424} or machine-to-machine payments \cite{8016254}.

In this context, the scope of this paper is to examine various blockchain platforms and technologies for the capability of processing mobile payments, management and settlements in vending machines. Based on these findings we propose a method by using a combination of multiple distributed ledger and blockchain technologies. In the further we implement a proof of concept by using a coffee machine as a special kind of vending machine. 